\documentclass[12pt]{report}
    \usepackage[utf8]{inputenc}
    \usepackage[spanish]{babel}
    \usepackage{amsmath}
    \usepackage{graphics}
    \usepackage{amssymb}
    \setlength{\oddsidemargin}{0.5cm}
    \setlength{\evensidemargin}{0.5cm}
    \setlength{\textwidth}{15cm}
    \setlength{\topmargin}{-2cm}
\begin{document}
\begin{center}
    \textsf{\Large Probabilidad 1}
    \par\medskip
    \textsf{\large Parcial 1}
\end{center}
\hrule
\par\bigskip

\begin{enumerate}
    \item Prueba que la intersección arbitraria de $\sigma$-álgebras es $\sigma$-álgebra.
    \item En una mesa circular se sientan 10 personas. ¿Cuál es la probabilidad de que los integrantes de una pareja en particular se siente uno junto al otro?
    \item Si $P(A) = 3/4$ y $P(B) = 3/8$, prueba que 
    \begin{itemize}
        \item $P(A \cup B) \geq 3/4$.
        \item $1/8 \leq P(A\cap B) \leq 3/8$.
    \end{itemize}
    \item Una panadería hornea 80 hogazas el día 10 de ellas están por debajo del peso requerido. Un inspector pesa 5 hogazas seleccionadas aleatoriamente. ¿Cuál es la probabilidad de que alguna hogaza baja en peso sea descubierta?
    \item Dados $P(A) = 1/3$, $P(B) = 1/4$ y $P(A\cap B) = 1/6$, encuentra: 
    
    $P(A^c)$, $P(A^c\cup B)$, $P(A\cup B^c)$, $P(A^c\cup B^c)$ y $P(A\cap B^c)$.
    
\end{enumerate}
\end{document}
