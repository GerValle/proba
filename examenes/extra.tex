\documentclass{report}
    \usepackage[utf8]{inputenc}
    \usepackage[spanish]{babel}
    \usepackage{amsmath}
    \usepackage{graphics}
    \usepackage{amssymb}
    \setlength{\oddsidemargin}{0.5cm}
    \setlength{\evensidemargin}{0.5cm}
    \setlength{\textwidth}{15cm}
    \setlength{\topmargin}{-2cm}
\begin{document}
\begin{center}
    \textsf{\Large Probabilidad 1}
    \par\medskip
    \textsf{\large Examen Extraordinario}
    \end{center}
    \hrule
    \par\bigskip

\begin{enumerate}
    \item Una mujer tiene n pares de zapatos en desorden. En un viaje intempestivo escoge al azar $2r$ zapatos ($r2\leq 2n$). Calcula la probabilidad de que en el conjunto escogido:
    \begin{enumerate}
        \item No haya ningún par completo.
        \item Haya exactamente un par completo.
        \item Haya $r$ pares completos.
    \end{enumerate}
    \item Demuestra que efectivamente la función de probabilidad binomial negativa dad por:
    $$
    f(x)= \binom{r+x-1}{x}p^r(1-p)^x
    $$
    para $x= 0,1,2,\ldots$, es efectivamente una función de probabilidad.
    %\item Una alumno debe vender 2 libros de una colección de 6 libros de matemáticas, 7 de ciencia y 4 de economía. De cuantas formas puede seleccionar los dos libros si:
    %\begin{enumerate}
    %    \item Ambos libros son de la misma materia
    %    \item Los libris son de diferente materia
    %\end{enumerate}
    %\item Una persona tiene 8 amigos, de los cuales 5 serán invitados a su fiesta. 
    %\begin{enumerate}
    %    \item Cuantas elecciones puede hacer si dos de sus amigos se detestan y no pueden ser invitados de forma conjunta?
    %    \item Cuantas elecciones puede hacer si existe una pareja que de ser inivitados, deben asistir ambos?
    %\end{enumerate}
    %\item Sean $E$, $F$ y $G$ eventos. Encuentra expresiones para las siguientes circunstancias:
    %\begin{enumerate}
    %    \item Solo ocurre $E$.
    %    \item Ocurren $E$ y $F$ pero no $G$.
    %    \item Al menos dos eventos ocurren.
    %    \item A lo mas dos eventos ocurren.
    %    \item Exactamente dos eventos ocurren.
    %\end{enumerate}
    %\item Muesta que la probabilidad de que exactamente uno de los eventos $E$ o $F$ ocurra es igual a $P(E)+P(F)-2P(E\cap F)$.
    \item Supongamos que en una población el 5\% de los hombres son portadores de cierta enfermedad y el 1\% de las mujeres también lo son. Se selecciona una persona al azar y se le hace una prueba para detectar la enfermedad. La prueba es positiva. ¿Cuál es la probabilidad de que la persona sea mujer? Supon que existen el mismo número de hombres y mujeres en la población. ¿Que pasa si la cantidad de hombres es el doble que la de mujeres?
    
    \item Sea $E$ independiente de $F$ y de $G$. Si $F\cap G = 0$, prueba que $E$ es independiente de $F\cup G$.
    %\item Sea $X$ es una variable aleatoria binomial con valor esperado 6 y varianza 2.4. Encuentra $P(X=5)$.
    \item El número de huevos que un insecto deposita en una hoja de una planta es una variable aleatoria con distribución Poisson con parámetro $\lambda$. Sin embargo, esta variable aleatoria puede ser observada solo si es positiva ya que si el número de huevos es cero, no podemos saber si el insecto estuvo presente o no. Si $Y$ es el número de huevos observados, entonces:
    $$
    P(Y=k)=P(X=i\vert X > 0)
    $$
    donde $X$ es una variable aleatoria Poisson con parámetro $\lambda$. Encuentra $E[Y]$.
 
    %\item Sean $f_1(x)$ y $f_2(x)$ funciones de densidad de probabilidad, $\theta_1 > 0$ y $\theta_2 > 0$ constantes tales que $\theta_1 + \theta_2 = 1$. Si $f(x) = \theta_1 f_1(x) + \theta_2f_2(x)$, muestre que $f(x)$ es entonces una función de densidad.

    %\item Sea $X$ la diferencia entre el número de soles y el número de águilas obtenidos mediante $n$ volados de una moneda justa.
    %      \begin{enumerate}
    %          \item ¿Cuáles son los valores posibles de $X$?
    %          \item Para n = 3, calcule la f. m. p. de X.
    %      \end{enumerate}

    %\item De los consumidores de una estación de gas 30 \% seleccionan gasolina regular, 20 \% premium y 50 \% diesel. De 100 de los siguientes consumidores, ¿cuál es la media y la varianza del número de consumidores que seleccionan gasolina regular?

    %\item Sea X una v.a. con distribución Uniforme continua en el intervalo $(1, a)$, con $a > 1$. Si $E[X] = 6 Var(X)$, calcule el valor de $a$.

    %\item El costo inicial de cierta máquina es \$3. El tiempo de vida de dicha máquina tiene una distribución exponencial con media de 3 años. El fabricante está considerando ofrecer una garantía, la cual paga \$3 si la máquina se descompone durante el primer año, paga \$2 si se descompone durante el segundo año y paga \$1 si se descompone durante el tercer año (si la máquina se descompone después del tercer año, no paga nada). Calcule el pago esperado de la garantía.

    %\item Sea $X\sim N(\mu,\sigma^2)$. Encuentra el tercer ($E\left[X^3\right]$) y  cuarto momento ($E\left[X^4\right]$)  de $X$

    %\item Demuestra que si $X\sim N(0,1)$ entonces $Y = \sigma X + \mu \sim N(\mu,\sigma^2) $.
    \item Muestra que la densidad de $X\sim N(\mu,\sigma^2)$ es simétrica al rededor de su media

    %\item Sea $X$ una variable aleatoria con densidad:
    %      $$
    %          f(x) = c\frac{1}{\sqrt{2\pi}}e^{-x^2/2}, \qquad \text{ para } x>0.
    %      $$
    %      Encuentra $c$ y calcula el valor esperado de $X$.
    %\item Sea $X$ una variable aleatoria con densidad:
    %      $$
    %          f(x)=c(1-x^2) \text{ para } -1 < x <1
    %      $$
    %      \begin{enumerate}
    %          \item Encuentra $c$.
    %          \item ¿Cuál es la función acumulada de $X$?
    %      \end{enumerate}
    %\item Sea $X$ una variable aleatoria con densidad:
    %      $$
    %          f(x)= a+bx^2 \text{ para }0\leq x\leq 1
    %      $$
    %      Si $E[X]=3/5$ encuentra $a$ y $b$.
    %\item Si $X\sim N(10, 36)$, calcula:
    %      \begin{itemize}
    %          \item $P(X > 5)$
    %          \item $P(4 < X < 16)$
    %          \item $P(X<8)$
    %          \item $P(\vert X\vert > 5)$
    %      \end{itemize}
    %\item El número de años que una radio funciona se destribuye de forma exponencial con parámetro $\lambda=1/8$. Si Juan compra una radio usada, ¿cuál es la probabilidad de que siga funcionando despuede de 8 años adicionales?



\end{enumerate}
\end{document}