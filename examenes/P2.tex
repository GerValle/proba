\documentclass[12pt]{report}
    \usepackage[utf8]{inputenc}
    \usepackage[spanish]{babel}
    \usepackage{amsmath}
    \usepackage{graphics}
    \usepackage{amssymb}
    \setlength{\oddsidemargin}{0.5cm}
    \setlength{\evensidemargin}{0.5cm}
    \setlength{\textwidth}{15cm}
    \setlength{\topmargin}{-2cm}
\begin{document}
\begin{center}
    \textsf{\Large Probabilidad 1}
    \par\medskip
    \textsf{\large Parcial 2}
\end{center}
\hrule
\par\bigskip

\begin{enumerate}
    \item Tomás entra a su cuarto a oscuras y selecciona aleatoriamente una prenda de una cómoda con dos cajones. Uno de los cajones tiene 6 calcetines y 6 camisetas y el otro cajón tiene 2 calcetines y tres calzones. ¿Cuál es la probabilidad de que la prenda que ha seleccionado sea un calcetín?
    \item Suponiendo que $P(B)\neq 0$ y $P(B^c)\neq 0$, muestra que si $P(A\vert B) \geq P(A)$ entonces $P(A\vert B^c) \leq P(A)$
    \item Supongamos que la función de distribución de $X$ esta dada por
          \begin{equation*}
              F(b)=
              \begin{cases}
                  0                         & b < 0      \\
                  \frac{b}{4}               & 0\leq b <1 \\
                  \frac{1}{2}+\frac{b-1}{4} & 1\leq b <2 \\
                  \frac{11}{12}             & 2\leq b <3 \\
                  1                         & 3\leq b
              \end{cases}
          \end{equation*}
          \begin{enumerate}
              \item Encuentra $P\left(X=i\right)$ para i=1,2,3.
              \item Encuentra $P\left(\frac{1}{2}<X<\frac{3}{2}\right)$.
          \end{enumerate}
    \item Si $E\left[X\right]=1$ y $Var\left(X\right)=5$, encuentra:
          \begin{enumerate}
              \item $E\left[(2+X)^2\right]$.
              \item $Var\left(4+3X\right)$
          \end{enumerate}
    \item Tiramos dos dados. Si $X$ es la diferencias de sus caras:
          \begin{enumerate}
              \item Exhibe la función de probabilidad de masa de $X$.
              \item Calcula $E\left[X\right]$.
              \item Calcula $Var\left(X\right)$.
          \end{enumerate}
\end{enumerate}
\end{document}