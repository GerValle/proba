\documentclass[12pt]{report}
    \usepackage[utf8]{inputenc}
    \usepackage[spanish]{babel}
    \usepackage{amsmath}
    \usepackage{graphics}
    \usepackage{amssymb}
    \setlength{\oddsidemargin}{0.5cm}
    \setlength{\evensidemargin}{0.5cm}
    \setlength{\textwidth}{15cm}
    \setlength{\topmargin}{-2cm}
\begin{document}
\pagestyle{empty}
\begin{center}
    \textsf{\Large Probabilidad 1}
    \par\medskip
    \textsf{\large Parcial 1}
\end{center}
\hrule
\par\bigskip

\begin{enumerate}
    \item En una mesa circular se sientan 10 personas. ¿Cuál es la probabilidad de que los integrantes de una pareja en particular se siente uno junto al otro?
    %\item Una panadería hornea 80 hogazas el día 10 de ellas están por debajo del peso requerido. Un inspector pesa 5 hogazas seleccionadas aleatoriamente. ¿Cuál es la probabilidad de que alguna hogaza baja en peso sea descubierta?
    \item ¿Cuantas señales se pueden formar con 10 banderas si 5 de ellas son rojas, 3 verdes y 2 amarillas? (banderas del mismo color son indistinguibles entre sí.)
    \item Un estudiante debe vender 2 libros de una colección de 6 libros de cálculo, 7 de álgebra y 5 de probabilidad. ¿De cuantas maneras puede hacerlo si:
    \begin{itemize}
        \item Los libros deben ser de la misma materia?
        \item Los libros deben ser de materias distintas?
    \end{itemize}
    (Por materia todos los libros son dierentes)
    %\item Dados $P(A) = 1/3$, $P(B) = 1/4$ y $P(A\cap B) = 1/6$, encuentra: 
    %$P(A^c)$, $P(A^c\cup B)$, $P(A\cup B^c)$, $P(A^c\cup B^c)$ y $P(A\cap B^c)$.
    \item Para $n>0$ prueba que:
    $$
    \sum_{i=0}^n (-1)^i\binom{n}{i} = 0
    $$
    \emph{Hint:} Usa el teorema del binomio.
    
\end{enumerate}
\end{document}