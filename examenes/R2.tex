\documentclass[12pt]{report}
    \usepackage[utf8]{inputenc}
    \usepackage[spanish]{babel}
    \usepackage{amsmath}
    \usepackage{graphics}
    \usepackage{amssymb}
    \setlength{\oddsidemargin}{0.5cm}
    \setlength{\evensidemargin}{0.5cm}
    \setlength{\textwidth}{15cm}
    \setlength{\topmargin}{-2cm}
\begin{document}
\begin{center}
    \textsf{\Large Probabilidad 1}
    \par\medskip
    \textsf{\large Reposición 2}
\end{center}
\hrule
\par\bigskip

\begin{enumerate}
    \item Un dado justo tiene las caras 1, 2 y 3 pintadas de verde y las caras 4,5 y 6 coloreadas de rojo. 

    Supongamos que el dado esta cargado  de modo que:
    \begin{equation*}
        P(1)=P(3)=P(5)=\frac{1}{9} \text{ y } P(1)=P(3)=P(5)=\frac{2}{9.}
    \end{equation*}
    Después de lanzar el dado:
    \begin{enumerate}
        \item ¿Cuál es la probabilidad de que el número sea par?
        \item Si lo único que logras ver, es que la cara que muestra el dado es verde. ¿Cuál es la probabilidad de que la cara sea par?
    \end{enumerate}

    \item Suponiendo que $P(B)\neq 0$ y $P(B^c)\neq 0$, muestra que si $P(A\vert B) \geq P(A)$ entonces $P(A\vert B^c) \leq P(A)$
    \item Supongamos que la función de distribución de $X$ esta dada por
          \begin{equation*}
              F(b)=
              \begin{cases}
                  0                         & b < 0      \\
                  \frac{b}{4}               & 0\leq b <1 \\
                  \frac{1}{2}+\frac{b-1}{4} & 1\leq b <2 \\
                  \frac{11}{12}             & 2\leq b <3 \\
                  1                         & 3\leq b
              \end{cases}
          \end{equation*}
          \begin{enumerate}
              \item Encuentra $P\left(X=i\right)$ para i=1,2,3.
              \item Encuentra $P\left(\frac{1}{2}<X<\frac{3}{2}\right)$.
          \end{enumerate}
    \item Si $E\left[X\right]=1$ y $Var\left(X\right)=5$, encuentra:
          \begin{enumerate}
              \item $E\left[(2+X)^2\right]$.
              \item $Var\left(4+3X\right)$
          \end{enumerate}
    \item Tiramos dos dados. Si $X$ es el producto de sus caras:
          \begin{enumerate}
              \item Exhibe la función de probabilidad de masa de $X$.
              \item Calcula $Var\left(X\right)$.
              \item Calcula $P(\vert X\vert > 2)$.
              \item Calcula $P(\vert X\vert > 4\vert X>2)$.

          \end{enumerate}
\end{enumerate}
\end{document}