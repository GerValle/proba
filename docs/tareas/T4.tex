\documentclass{article}
\usepackage[spanish]{babel}
\usepackage[utf8]{inputenc}
\usepackage{amsmath,amssymb}
\usepackage[a4paper, total={6in, 10in}]{geometry}
\begin{document}
\begin{center}
  \textsf{\Large Probabilidad 1}
  \par\medskip
  \textsf{\large Tarea 4}
\end{center}
\hrule
\par\bigskip

\begin{enumerate}
  \item Tiramos dos dados justos. Sea $X$ igual al producto de los dos dados. Calcula $P\left(X=i\right)$, para $i=1, 2, \ldots, 36.$
  \item Lanzamos un dado dos veces. ¿Cuales son los posibles valores que las siguientes variables aleatorias pueden tomar?
        \begin{enumerate}
          \item El valor máximo que puede salir en los dos lanzamientos.
          \item El valor mínimo que puede salir en los dos lanzamientos.
          \item La suma de los dos lanzamientos.
          \item El valor del primer lanzamiento menos el valor del segundo.
        \end{enumerate}
  \item Supongamos que la función de distribución de $X$ esta dada por
        \begin{equation*}
          F(b)=
          \begin{cases}
            0                         & b < 0      \\
            \frac{b}{4}               & 0\leq b <1 \\
            \frac{1}{2}+\frac{b-1}{4} & 1\leq b <2 \\
            \frac{11}{12}             & 2\leq b <3 \\
            1                         & 3\leq b
          \end{cases}
        \end{equation*}
        \begin{enumerate}
          \item Encuentra $P\left(X=i\right)$ para i=1,2,3.
          \item Encuentra $P\left(\frac{1}{2}<X<\frac{3}{2}\right)$.
        \end{enumerate}
  \item Si $E\left[X\right]=1$ y $Var\left(X\right)=5$, encuentra:
        \begin{enumerate}
          \item $E\left[(2+X)^2\right]$.
          \item $Var\left(4+3X\right)$
        \end{enumerate}
  \item Tiramos dos dados. Si $X$ es el número mas grande en las caras mostradas, calcula $P\left(\left\{X\in[2,4]\right\}\right)$ .
  \item Muestra que si $X$ es una variable aleatoria con función de probabilidad de masa:
        $$
          P(X =i) = p(i), \quad i = 0, 1, 2,\ldots
        $$
        es decir $S = \left\{0, 1, 2, \ldots\right\}$. Entonces:
        $$
          E[X] = \sum_{i=1}^\infty P(X\geq i).
        $$
        De manera mas general, para valores no negativos $a_j$, $j\geq 1$, muestra que:
        $$
        \sum_{j=1}^\infty(a_1+\ldots + a_j)P(X=j) = \sum_{j=1}^\infty a_jP(X\geq j).
        $$
        adicionalmente muestra que:
        $$
        E[X(X+1)] = 2\sum_{i=1}^\infty iP(X\geq i).
        $$

  \item Tiramos dos dados. Si $X$ es la diferencias de sus caras:
        \begin{enumerate}
          \item Exhibe la función de probabilidad de masa de $X$.
          \item Calcula $E\left[X\right]$.
          \item Calcula $Var\left(X\right)$.
          \item Calcula $P(\vert X\vert > 2)$.
          \item Calcula $P(\vert X\vert > 2\vert X>0)$.
          \item Calcula la probabilidad de que $X$ sea par.
        \end{enumerate}
  \item Una urna contiene 7 bolas blancas numeradas: $1,2,\ldots, 7$ y 3 bolas negras numeradas $8,9,10$. Seleccionamos 5 bolas aleatoriamente, (a) con reemplazo, (b) sin reemplazo. Para cada uno de los casos (a) y (b), da la función de probabilidad de masa de:
        \begin{enumerate}
          \item El número de bolas blancas seleccionadas.
          \item El número mínimo en la muestra.
          \item El número máximo en la muestra.
        \end{enumerate}
        \textbf{Hints:} En cada uno de los incisos identifica bien los valores que puede tomar la variable aleatoria correspondiente. Para el inciso (b) define $X_{min}$ como el número mínimo en la muestra, calcula $P(X_{min} > k)$ y prueba que 
        $$
        P(X_{min} = k) = P(X_{min} > k-1) - P(X_{min} > k).
        $$
        Para el inciso (c) define $X_{max}$ como el número máximo en la muestra, calcula $P(X_{max} \leq k)$ y determina $P(X_{max} = k)$ en términos de $P(X_{max} \leq k)$ y $P(X_{max} \leq k-1)$..
  \item Con los datos del problema anterior, calcula la función de probabilidad de masa del número mínimo de selecciones necesarias para obtener una bola blanca.
\end{enumerate}

\end{document}
