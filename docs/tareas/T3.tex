\documentclass{report}
    \usepackage[utf8]{inputenc}
    \usepackage[spanish]{babel}
    \usepackage{amsmath}
    \usepackage{graphics}
    \usepackage{amssymb}
    \setlength{\oddsidemargin}{0.5cm}
    \setlength{\evensidemargin}{0.5cm}
    \setlength{\textwidth}{15cm}
    \setlength{\topmargin}{-2cm}
\begin{document}
\begin{center}
    \textsf{\Large Probabilidad 1}
    \par\medskip
    \textsf{\large Tarea 3}
    \end{center}
    \hrule
    \par\bigskip

\begin{enumerate}
    \item Un dado justo tiene las caras 1, 2 y 3 pintadas de verde y las caras 4,5 y 6 coloreadas de rojo. Después de lanzar el dado:
    \begin{enumerate}
        \item ¿Cuál es la probabilidad de que el número sea par?
        \item Si lo único que logras ver, es que la cara que muestra el dado es verde. ¿Cuál es la probabilidad de que la cara sea par?
    \end{enumerate}
    Supongamos que el dado esta cargado  de modo que:
    \begin{equation*}
        P(1)=P(3)=P(5)=\frac{1}{9} \text{ y } P(1)=P(3)=P(5)=\frac{2}{9.}
    \end{equation*}
    \begin{enumerate}
        \item ¿Cuál es la probabilidad de que el número sea par?
        \item Si lo único que logras ver, es que la cara que muestra el dado es verde. ¿Cuál es la probabilidad de que la cara sea par?
    \end{enumerate}
    \item Dos dados justos son lanzados y el resultado es mantenido en secreto. 
    \begin{enumerate}
        \item ¿Cuál es la probabilidad de que la suma sea al menos 5?
        \item Si se sabe que al menos uno de los dados es 1. ¿Cuál es la probabilidad de que la suma sea al menos 5?
    \end{enumerate}
    \item Tomás entra a su cuarto a oscuras y selecciona aleatoriamente una prenda de una cómoda con dos cajones. Uno de los cajones tiene 6 calcetines y 6 camisetas y el otro cajón tiene 2 calcetines y tres calzones. ¿Cuál es la probabilidad de que la prenda que ha seleccionado sea un calcetín?   
    \item Muestra que las siguientes condiciones son equivalentes:
    \begin{enumerate}
        \item $A$ es independiente de cualquier evento $B\subset \Omega$.
        \item $P(A)=0$ o $P(A)=1$. 
    \end{enumerate}
    \item Lanzamos una moneda justa tres veces. Considera los siguientes eventos:
    \begin{align*}
        A & = \text{El lanzamiento 1 y el lanzamiento 2 producen diferentes resultados}\\
        B & = \text{El lanzamiento 2 y el lanzamiento 3 producen diferentes resultados}\\
        C & = \text{El lanzamiento 3 y el lanzamiento 1 producen diferentes resultados}\\
    \end{align*}
    Muestra que $P(A)=P(A\vert B)$ y $P(A)=P(A\vert C)$ sin embargo $P(A)\neq P(A\vert B\cap C)$.

    Este problema muestra que $A, B$ y $C$ son independientes dos a dos, sin embargo si consideramos a los tres juntos no podemos asegurar independencia.

    En general:  $A_1,\ldots, A_n$ son independientes si $P(A_1\cap\cdots\cap A_n)=P(A_1)\cdots P(A_n)$.
\end{enumerate} 
\end{document}