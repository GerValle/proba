% Format: Latex
% Encoding: UTF-8
% Tarea 1 de probabilidad 1
% UNAM
% Facultad de Ciencias
\documentclass[12pt]{extreport}
    \usepackage[utf8]{inputenc}
    \usepackage[spanish]{babel}
    \usepackage{fourier}   
    \usepackage{amsmath}
    \usepackage{graphics}
    \usepackage{amssymb}
    \setlength{\oddsidemargin}{0.5cm}
    \setlength{\evensidemargin}{0.5cm}
    \setlength{\textwidth}{15cm}
    \setlength{\topmargin}{-2cm}
\begin{document}
\begin{center}
    \textsf{\Large Probabilidad 1}
    \par\medskip
    \textsf{\large Tarea 1}
    \end{center}
    \hrule
    \par\bigskip

\begin{enumerate}
    \item Sea $X$ un conjunto y $\mathcal{E}$ la colección de subconjuntos:
    $$
    \mathcal{E} = \left\{A\subset X \mid A \text{ es contable o } A^c \text{ es contable}  \right\}
    $$
    Muestra que $\mathcal{E}$ es una $\sigma$-álgebra de subconjuntos de $X$.
    %\item Muestra que una $\sigma$-álgebra es cerrada bajo intersecciones numerables.
    \item Prueba que $\left|P(A)-P(B)\right|\leq P(A\triangle B)$.
    \item Prueba que $P(A\triangle C)\leq P(A\triangle B) + P(B\triangle C)$.
    \item Demuestra que si $P(A) = P(B) = 0 $ entonces $P(A\cup B) = 0$.
    \item Muestra que si $P(A) = P(B) = 1$ entonces $P(A\cap B) = 1$.
    \item Supongamos que $P(A)= 1/2$ y $P(B)= 2/3$. Muestra que:
    $$
    \frac{1}{6} \leq P(A\cap B) \leq \frac{1}{2}
    $$
    Da ejemplos que muestren que los valores extremos, $1/6$ y $1/2$ pueden obtenerse.
    \item Supongamos que $P(A) = 1/2$ y $P(B) = 1/3$. Muestra que
    $$
    \frac{1}{2}\leq P(A\cup B)\leq \frac{5}{6}
    $$
    Da ejemplos que muestren que los valores extremos, $1/2$ y $5/6$ pueden obtenerse.

    \item Sabemos que si los eventos $A$ y $B$ ocurren, entonces el evento $C$ también ocurre. Muestra que:
    $$
    P(C) \geq P(A) + P(B) - 1
    $$ 
    \item Sea $A$ un evento y $B$ un evento tal que $P(B)>0$. Muestra que:
\end{enumerate}
\end{document}