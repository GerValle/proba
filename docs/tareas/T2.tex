\documentclass{report}
    \usepackage[utf8]{inputenc}
    \usepackage[spanish]{babel}
    \usepackage{amsmath}
    \usepackage{graphics}
    \usepackage{amssymb}
    \setlength{\oddsidemargin}{0.5cm}
    \setlength{\evensidemargin}{0.5cm}
    \setlength{\textwidth}{15cm}
    \setlength{\topmargin}{-2cm}
\begin{document}
\begin{center}
    \textsf{\Large Probabilidad 1}
    \par\medskip
    \textsf{\large Tarea 2}
    \end{center}
    \hrule
    \par\bigskip

\begin{enumerate}
    \item Diez personas se sientan aleatoriamente al rededor de una mesa
      redonda. Cual es la probabilidad de que una pareja particular se
      siente uno junto al otro?
    \item ¿Cuál es la probabilidad de obtener la palabra ABRACADABRA si las letras A, A, A, A, A, B, B, C, D, R son seleccionadas al azar? 
        \item Una panadería hornea 80 hogazas el día 10 de ellas están por debajo del peso requerido. Un inspector pesa 5 hogazas seleccionadas aleatoriamente. ¿Cuál es la probabilidad de que alguna hogaza baja en peso sea descubierta?
        \item En un juego de poker, cuál es la probabilidad de que:
        \begin{enumerate}
            \item Una mano consista de un, diez, un AS, una joto, un reina y un rey del mismo palo?
            \item Contenga 4 cartas de la misma denominación?
            \item Consista de cartas de denominación consecutiva exceptuando aces, jotos, reinas y reyes?
        \end{enumerate} 
        \item Seleccionamos 5 letras aleatoriamente, (a) con reemplazo, (b) sin reemplazo. Encuentra para cada uno de los casos (a) y (b) la probabilidad de que la palabra formada: (i) contenga una sola ``a", (ii) este formada por vocales, (iii) sea la palabra ``mujer".  
\end{enumerate}
\end{document}