\documentclass[12pt]{report}
    \usepackage[utf8]{inputenc}
    \usepackage[spanish]{babel}
    \usepackage{amsmath}
    \usepackage{graphics}
    \usepackage{amssymb}
    \setlength{\oddsidemargin}{0.5cm}
    \setlength{\evensidemargin}{0.5cm}
    \setlength{\textwidth}{15cm}
    \setlength{\topmargin}{-2cm}
\begin{document}
\begin{center}
    \textsf{\Large Probabilidad 1}
    \par\medskip
    \textsf{\large Tarea 2}
    \end{center}
    \hrule
    \par\bigskip

\begin{enumerate}
    %\item Diez personas se sientan aleatoriamente al rededor de una mesa
    %  redonda. Cual es la probabilidad de que una pareja particular se
    %  siente uno junto al otro?
    \item ¿Cuál es la probabilidad de obtener la palabra ABRACADABRA si las letras A, A, A, A, A, B, B, C, D, R, R son seleccionadas al azar? 
        \item Una panadería hornea 80 hogazas el día 10 de ellas están por debajo del peso requerido. Un inspector pesa 5 hogazas seleccionadas aleatoriamente. ¿Cuál es la probabilidad de que alguna hogaza baja en peso sea descubierta?
        \item En un juego de poker, cuál es la probabilidad de que:
        \begin{enumerate}
            \item Una mano consista de un, diez, un AS, una joto, un reina y un rey del mismo palo?
            \item Contenga 4 cartas de la misma denominación?
            \item Consista de cartas de denominación consecutiva exceptuando aces, jotos, reinas y reyes?
        \end{enumerate} 
        \item Seleccionamos 5 letras aleatoriamente, (a) con reemplazo, (b) sin reemplazo. Encuentra para cada uno de los casos (a) y (b) la probabilidad de que la palabra formada: (i) contenga una sola ``a", (ii) este formada por vocales, (iii) sea la palabra ``mujer".  
        \item Pedro, Hugo, Paco y Luis formaron una banda que consiste de 4 instrumentos. Si todos ellos pueden tocar todos los instrumentos, ¿de cuantas maneras pueden formar una banda?¿Que pasa si solo Pedro y Hugo pueden tocar todos los instrumentos y Paco y Luis solo pueden tocar la guitarra y el bajo?
        \item Un niño tiene 12 bloques, de los cuales 4 son negros, 4 son rojos, 3 son blancos y uno es azul. ¿Cuantos arreglos son posibles con estos bloques?
        \item De cuantas maneras podemos sentar a 8 personas en una fila si:
        \begin{enumerate}
            \item No existen restricciones para sentarlos
            \item Las personas $A$ y $B$ deben sentartes una junto a la otra.
            \item Tenemos 4 hombres y 4 mujeres y no podemos sentar a dos hombres o dos mujeres juntos.
            \item Hay 5 hombres y deben sentarse juntos
            \item Tenemos 4 parejas de casados y cada pareja debe sentarse junta.
        \end{enumerate}
        \item Vamos a repartir siete regalos entre cinco niños. De cuantas formas se pueden repartir los regalos si no es posible otorgarle mas de una regalo a un niño.
        \item Tenemos una urna con $N$ bolas, de las cuales $b$ de ellas con blancas y $r$ de ellas con rojas. Si extraemos $n$ bolas, 
        \begin{enumerate}
            \item ¿Cuál es la probabilidad de que exactamente $m$ de ellas sean blancas?
            \item ¿Cuál es la probabilidad de que al menos $m$ de ellas sean blancas? 
        \end{enumerate}
        \item ¿Cual es la probabilidad de que en un grupo con $n$ personas, al menos dos de ellas cumplan años el mismo día? ¿Cuantas personas deben estar presentes para que la probabilidad exceda el 90\%?
        
\end{enumerate}
\end{document}