\documentclass{report}
    \usepackage[utf8]{inputenc}
    \usepackage[spanish]{babel}
    \usepackage{amsmath}
    \usepackage{graphics}
    \usepackage{amssymb}
    \usepackage{dsfont}
    \setlength{\oddsidemargin}{0.5cm}
    \setlength{\evensidemargin}{0.5cm}
    \setlength{\textwidth}{15cm}
    \setlength{\topmargin}{-2cm}
\begin{document}
\begin{center}
    \textsf{\Large Probabilidad 1}
    \par\medskip
    \textsf{\large Tarea 1}
\end{center}
    \hrule
    \par\bigskip
% comentario

\begin{enumerate}
    \item Determina el número de maneras de poner 3 torres en un tablero de 5 x 5 sin que dos de ellas se ataquen.
    
    \item Una araña tiene 8 pies, 8 zapatos distintos y 8 calcetines distintos. Encuentra el número de maneras en las que puede ponerse los 8 calcetines y los 8 zapatos (considerando el orden en el que se los pone) con la condición de que antes de ponerse un zapato tiene que tener ya un calcetín en ese pie

    \item Demuestra que si $n \geq k \geq r \geq s$, entonces 
    \[
       \binom{n}{k}\binom{k}{r}\binom{r}{s} = \binom{n}{s}\binom{n-s}{r-s}\binom{n-r}{k-r} %= \frac{n!}{(n - r)!}
    \]

    \item Se llama mano de domino a cualquier colección de 7 de las 28 fichas. Se llama ficha doble a la ficha en que los dos números mostrados son iguales. ¿Cuántas manos de domino existen? ¿Cuántas manos de domino tiene por lo menos dos fichas dobles? 

    \item De un conjunto de 20 playeras de distintos colores se quieren escoger 5 de tal manera que 3 sean para vacaciones y 2 sean para la escuela. ¿De cuántas formas distintas es posible hacer la elección?

    \item Una baraja de poker tiene 13 símbolos y 4 palos (pijas, corazón, diamante y trébol)
    \begin{enumerate}
        \item ¿Cuántas manos de poker tienen exactamente un tercia (tres cartas del mismo nùmero y dos cartas con numero distinto de la tercia)
        \item ¿Cuántas manos de poker tienen corrida? (cinco cartas con numeración consecutiva)
    \end{enumerate}

    \item ¿Cuántas oraciones distintas se pueden escribir revolviendo los caracteres de la oración
    
    \begin{center} 
    ``Nunca subestimes el poder del interés compuesto''
    \end{center}
\end{enumerate}    
    
\end{document}