\documentclass[14pt]{extbook}
\usepackage[utf8]{inputenc}
\usepackage[spanish]{babel}
\usepackage{amsmath, amssymb}
\usepackage{mathpazo}


\begin{document}


\chapter{Teoría de conjuntos.}

La teoría de probabilidad tiene sus fundamentos en la teoría de conjuntos. Es mucho lo que podemos construir partiendo de esta base. Por el momento postergaremos la construcción de las ideas elementales de la probabilidad y nos concentraremos en proveer un repaso de algunos conceptos básicos de teoría de conjuntos.

No pretendemos dar una definición formal de conjunto, sin embargo usaremos el concepto libremente. 


Usualmente usaremos mayúsculas para referirnos a conjuntos y minúsculas para dentar elementos. Escribimos por ejemplo
$$
a \in A 
$$
para decir que $a$ es un elemento del conjunto $A$. Asimismo, usaremos mayúsculas caligráficas para denotar colecciones de conjuntos. Por ejemplo:
$$
A\in \mathcal{A}
$$
indica que el conjunto $A$ es un elemento de la colección $\mathcal{A}$.
Particularmente, dado un conjunto $A$ denotamos con $2^A$ a la \emph{potencia} de $A$, recordemos que la potencia de un conjunto es la colección de todos los posibles subconjuntos del mismo, de modo que $2^A$ es la colección de todos los posibles subconjuntos de $A$. Por otro lado, dado un conjunto $A$ denotaremos con $\vert A \vert$ a la cardinalidad de $A$ y con $2^{\vert A \vert}$ a la cardinalidad de su potencia.


Decimos que un conjunto $A$ esta \emph{contenido} en un conjunto $B$ si $x\in A$ entonces $x \in B$, escribimos entonces que:
$$
A\subset B
$$

Dos conjuntos son iguales si y solo si contienen los mismos elementos. Podemos expresarlo de otro modo; $A = B$ si y solo si $A\subset B$ y $B \subset A$.

La \emph{union} de los conjuntos $A$ y $B$ es el conjunto de todos los elementos que pertenecen a $A$ o pertenecen a $B$:
$$
A\cup B := \left\{x: x \in A \text{ o } x \in B\right\}
$$
de manera general, dada la colección $\mathcal{A} = \left\{A_i\right\}_{i\in I}$, donde $I$ es un conjunto de indices, entonces:
$$
\bigcup_{i\in I}A_i := \left\{x: x \in A_i \text{ para algún } i\in I \right\}
$$
$$
\bigcup_{i\in I}A_i := \left\{x: x \in A \text{ para algún } A\in \mathcal{A} \right\}
$$
o bien:


\end{document}