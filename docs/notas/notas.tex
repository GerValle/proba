\documentclass[14pt]{extbook}
\usepackage[utf8]{inputenc}
\usepackage[spanish]{babel}
\usepackage{amsmath, amssymb}
\usepackage{mathpazo}


\begin{document}


\chapter{Teoría de conjuntos.}

La teoría de probabilidad tiene sus fundamentos en la teoría de conjuntos. Es mucho lo que podemos construir partiendo de esta base. Por el momento postergaremos la construcción de las ideas elementales de la probabilidad y nos concentraremos en proveer un repaso de algunos conceptos básicos de teoría de conjuntos.

No pretendemos dar una definición formal de conjunto, sin embargo usaremos el concepto libremente. 



\end{document}