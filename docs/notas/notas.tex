\documentclass[14pt]{extreport}
\usepackage[utf8]{inputenc}
\usepackage[spanish]{babel}
\usepackage{microtype}
\setlength{\parskip}{1em}
%\usepackage{lmodern}
\usepackage{amssymb,amsmath}
\usepackage{mathpazo}
\usepackage{tikz}
\usetikzlibrary{babel}
\usetikzlibrary {positioning}

\usepackage[pdftex,
            pdfauthor={German Valle Trujillo},
            pdftitle={Probabilidad},
            pdfsubject={Probabilidad}]{hyperref}

\newtheorem{teo}{Teorema}



\begin{document}


\chapter{Teoría de conjuntos.}

La teoría de probabilidad tiene sus fundamentos en la teoría de conjuntos. Es mucho lo que podemos construir partiendo de esta base. Por el momento postergaremos la construcción de las ideas elementales de la probabilidad y nos concentraremos en proveer un repaso de algunos conceptos básicos de teoría de conjuntos.

No pretendemos dar una definición formal de conjunto, sin embargo usaremos el concepto libremente. 


Usualmente usaremos mayúsculas para referirnos a conjuntos y minúsculas para dentar elementos. Escribimos por ejemplo
$$
a \in A 
$$
para decir que $a$ es un elemento del conjunto $A$. Asimismo, usaremos mayúsculas caligráficas para denotar colecciones de conjuntos. Por ejemplo:
$$
A\in \mathcal{A}
$$
indica que el conjunto $A$ es un elemento de la colección $\mathcal{A}$.
Particularmente, dado un conjunto $A$ denotamos con $2^A$ a la \emph{potencia} de $A$, recordemos que la potencia de un conjunto es la colección de todos los posibles subconjuntos del mismo, de modo que $2^A$ es la colección de todos los posibles subconjuntos de $A$. Por otro lado, dado un conjunto $A$ denotaremos con $\vert A \vert$ a la cardinalidad de $A$ y con $2^{\vert A \vert}$ a la cardinalidad de su potencia.


Decimos que un conjunto $A$ esta \emph{contenido} en un conjunto $B$ si $x\in A$ entonces $x \in B$, escribimos entonces que:
$$
A\subset B
$$

Dos conjuntos son iguales si y solo si contienen los mismos elementos. Podemos expresarlo de otro modo; $A = B$ si y solo si $A\subset B$ y $B \subset A$.

En los sucesivo, es conveniente suponer que existe un conjunto $\Omega$ \emph{universal}, de modo que todo elemento que consideremos, será elemento de $\Omega$ y todo conjunto será subconjunto de $\Omega$.

La \emph{union} de los conjuntos $A$ y $B$ es el conjunto de todos los elementos que pertenecen a $A$ o pertenecen a $B$:
$$
A\cup B := \left\{x \in \Omega: x \in A \text{ o } x \in B\right\}
$$
de manera general, dada la colección $\mathcal{A} = \left\{A_i\right\}_{i\in I}$, donde $I$ es un conjunto de indices, entonces:
$$
\bigcup_{i\in I}A_i := \left\{x \in \Omega: x \in A_i \text{ para algún } i\in I \right\}
$$
\noindent
o bien:
\noindent
$$
\bigcup_{A\in \mathcal{A}}A := \left\{x\in \Omega: x \in A \text{ para algún } A\in \mathcal{A} \right\}
$$
si el conjunto $I$ es a lo mas numerable entonces la union es una union numerable en otro caso la union es no numerable\footnote{Habitualmente, cuando la colección es numerable, $I\subset \mathbb{N}$ o bien $I\subset \mathbb{Z}$. Si $I$es no numerable entonces lo común es que $I\subset\mathbb{R}$ aunque en ciertas aplicaciones este conjunto podría ser bastante mas complejo.} . 

La \emph{intersección} de los conjuntos $A$ y $B$ es el conjunto:
$$
A\cap B = \left\{x\in \Omega: x \in A \text{ y } x \in B\right\}
$$
es decir $A\cap B$ es el conjunto de todos los elementos que pertenecen a $A$ y pertenecen a $B$. Análogamente, tal como hicimos con la union, dada la colección $\mathcal{A} = \left\{A_i\right\}_{i\in I}$:
$$
\bigcap_{i\in I}A_i := \left\{x\in \Omega: x \in A_i \text{ para todo } i\in I \right\}
$$
\noindent
o bien:
\noindent
$$
\bigcap_{A\in \mathcal{A}}A := \left\{x\in \Omega: x \in A \text{ para todo } A\in \mathcal{A} \right\}
$$
del mismo modo, si $I$ es a lo mas numerable entonces la intersección es numerable.

El \emph{conjunto vacío} es el conjunto que no contiene elementos y lo denotamos con $\emptyset$. Es claro que $\emptyset \subset A$ para cualquier conjunto $A$. Por otro lado decimos que $A$ y $B$ son \emph{disjuntos} si $A\cap B = \emptyset$. En general decimos que una colección de conjuntos $\mathcal{A} = \left\{A_i\right\}_{i\in I}$ es una colección de disjuntos \emph{dos a dos} si $A_i\cap A_j = \emptyset$ para todo $i, j\in I$ tales que $i\neq j$. Será en general una colección de disjuntos si $\cap_{j\in J} A_j = \emptyset$ para todo $J \subset I$ donde los elementos de $\left\{A_j\right\}_{j\in J}$ no son todos iguales.

Definimos la \emph{diferencia} de los conjuntos $A$ y $B$ como:
$$
A\setminus B = \left\{x\in \Omega: x  \in A \text{ y } x\notin B\right\}
$$
$A\setminus B$ también es conocido como el \emph{complemento relativo} de $B$ con respecto a $A$. De manera particular al conjunto 
$$
A^c = \Omega\setminus A = \left\{x\in \Omega: x\notin A\right\}
$$
lo conocemos como el \emph{complemento} de $A$.
Observemos de la definición de $A\setminus B$ que:
$$
A\setminus B = A \cap B^c
$$
Un derivado inmediato es la definición de la \emph{diferencia simétrica} de $A$ y $B$:
$$
A\triangle B = (A\setminus B) \cup (B\setminus A)
$$
la diferencia simétrica no es mas que el conjunto de todo aquellos elementos que pertenecen a $A$ o pertenecen a $B$ pero no a ambos. En este sentido, podemos reescribir:

$$
A\triangle B = (A\cup B) \setminus (B\cap A)
$$
La prueba de la equivalencia de las dos igualdades que hemos exhibido para $A\triangle B$ no la daremos, quedará como ejercicio, sin embargo la prueba tiene fundamento en la distribución de la unión e intersección de conjuntos, es decir, dados los conjuntos: $A$, $B$ y $C$ tenemos que:
$$
A \cap (B\cup C) = (A\cap B) \cup (A\cap C)
$$
y 
$$
A \cup (B\cap C) = (A\cup B) \cap (A\cup C)
$$
de hecho, esta propiedad puede generalizarse para colecciones mas grandes de conjuntos. Si $\left\{B_i\right\}_{i\in I}$ es una colección de conjuntos, entonces:
$$
A\cap \left(\cup_{i\in I} B_i\right) = \cup_{i\in I}\left(A\cap B_i\right)
$$
y 
$$
A\cup \left(\cap_{i\in I} B_i\right) = \cap_{i\in I}\left(A\cup B_i\right)
$$

El siguiente resultado es sin duda es uno de los más útiles de la teoría de conjuntos:
\begin{teo}[Ley de De Morgan]
Dada una colección de conjuntos $\left\{A_i\right\}_{i\in I}$ tenemos que:
$$
\left(\bigcup_{i\in I} A_i\right)^c =\bigcap_{i\in I} A_i^c
$$
y 
$$
\left(\bigcap_{i\in I} A_i\right)^c =\bigcup_{i\in I} A_i^c
$$
\end{teo}

\section{Funciones.}
Una relación $R$ es un subconjunto del producto cartesiano $A\times B$ de los conjuntos $A$ y $B$. Si $(a,b)\in R$ entonces decimos que $a$ esta relacionado con $b$ y escribimos $aRb$. Una \emph{función de $A$ a $B$} denotada con: $f:A\to B$ es una relación en la cual para cada $a\in A$ existe un único $b\in B$ tal que $aRb$. En este caso escribimos $f(a) = b$.

El conjunto:
$$
f(E) := \left\{f(x)\in B: x\in E\subset A\right\} 
$$
se conoce como \emph{imagen} de $E$ bajo $f$. Es claro que $f(E) \subset B$.

Decimos que una función es \emph{inyectiva} o ``\emph{uno a uno}'' si $f(a) = f(b)$ implica que $a=b$. Equivalentemente, si $a\neq b$ entonces $f(a)\neq f(b)$. Por otro lado, decimos que una función $f:A\rightarrow B$ es \emph{sobre} si para todo $b\in B$ existe $a\in A$ tal que $f(a)=b$. Notemos que en general si $f:A\rightarrow B$ entonces $f(A)\subset B$, pero si $f$ es sobre, entonces $f(A) = B$. Al conjunto $f(A)$ se le conoce como \emph{rango} de $f$. Por lo tanto una función $f:A\rightarrow B$ es sobre si $B$ es igual a su rango.

Otro concepto igualmente importante es de algún modo el opuesto a la imagen de un conjunto. Dada una función $f:A\rightarrow B$ definimos la \emph{preimagen} de un conjunto $Y \subset B$ como:
$$
f^{-1}(Y) = \left\{x\in A: f(x) \in Y\right\}
$$
de forma coloquial, el conjunto $f^{-1}(Y)\subset A$ son todos aquellos elementos de $A$ que bajo la función ``van a dar'' a $Y$. Hacemos énfasis en que $f^{-1}(Y)$ es un subconjunto de $A$ y no un elemento. Decimos que una función es \emph{biyectiva} si es a la vez uno a uno y sobre.

Decimos que un conjunto $I$ es \emph{a lo mas numerable}\footnote{Un conjunto numerable también se conoce como conjunto \emph{contable}} si es posible definir una función inyectiva, de los elementos de $I$ a algún subconjunto de los naturales. Si esto no es posible, entonces decimos que $I$ es \emph{no numerable}. Esta idea de algún modo se extiende si consideramos una colección $\left\{A_i\right\}_{i\in I}$. Esta colección sera a lo mas numerable si y solo si el índice $I$ es a lo mas numerable. Observemos que por definición, un conjunto finito es a lo mas numerable.
\end{document}

